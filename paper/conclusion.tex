\section{Conclusion and Future Work}

Test independence is broadly assumed but rarely addressed, and
the impact of test dependence on downstream
testing techniques is unclear and has largely been ignored in
previous software testing research. This paper describes
one of the first empirical studies on real-world test suites to
understand the impact of test dependence. The experimental
results show that test dependence \textit{does} affect the results
of well-known test prioritization, test selection, and test
parallelization techniques. To cope with the impact of test
dependence, this paper proposes a family of techniques
to enhance existing testing techniques, making them respect test dependence
and produce consistent testing results.


Our findings are suggestive to practitioners and researchers.
Both can learn that the impact of test dependence 
should not be ignored any longer. Researchers are posed important
but challenging problems in designing new testing techniques,
such as how to adapt testing methodologies to account for
dependent tests how to detect and nullify the impact
caused by dependent tests.

Our future work should focus on the following directions:

\vspace{1mm}

\noindent \textbf{{Larger experiments.}}
In this paper, we have shown that dependent tests
can compromise the application of
\prionum test prioritization, \selnum test selection,
and \parnum test parallelization techniques.
%assumption is not true. However,
We plan to conduct larger experiments and more
comprehensive evaluations to
measure the impact of dependent tests on other
downstream testing techniques, such as mutation testing~\cite{Zhang:2012:RMT, Schuler:2009:EMT, Zhang:2013:FMT},
test factoring~\cite{Saff:2005, Wu:2010:LRV}, and experimental
debugging techniques~\cite{Zeller:2002, Steimann:2013, Zhang:2013:IMF}.
We are also interested in developing techniques that can enhance these
testing techniques to respect test dependence.



\vspace{1mm}

\noindent \textbf{{Eliminating dependent tests.}}
%As reflected in our previous study~\cite{},
Another way to cope with the impact of test dependence
is developing techniques to eliminate dependent tests
before their impact arises.
However, the practice of eliminating dependent tests
remains mostly manual and ad hoc --- software developers
usually manually hardcode test
execution orders in a configuration file or
simply merge or remove tests.
A more flexible and robust methodology for
dependent test elimination should be developed.
%While our technique proposed in Section~\ref{}
%provides a first step in understanding the root cause
%of test dependence, the question
%of automatically removing dependent tests
%still remains open.
This question also applies to automated test generators,
and some work has been developed to alleviate
this problem~\cite{vmvm, RobinsonEPAL2011,fraseretal:ISSTA:2011}.




\begin{comment}

The topic of test dependence has been largely ignored in previous research on software testing, particularly its effects on testing techniques such as test prioritization. We believe we are the first to study the impact test dependence has on test prioritization through the application of test prioritization techniques on real-world programs. Other studies of applying test prioritization techniques to real-world programs include one from August 2000~\cite{} and another from October 2001~\cite{}. Although we concluded that the impact of dependent tests on test prioritization is minimal, we showed that test dependence does indeed affect the execution results of test prioritization techniques. We were able to show this by designing and implementing five test prioritization techniques and applying them to five real-world programs. Lastly, we described impending features to our set of existing tools that when generating test prioritization execution orders will take into account which tests depend on one another. 

Our future work should focus on the following directions:
\begin{itemize}
\item Understand the impact test dependence may have on other testing techniques. Dependent tests can compromise the application of testing techniques such as test generation, selection, prioritization, and parallelization, since most current testing techniques just assume independence and make no statement about what happens when this assumption is not true [1]. In this paper we have addressed how test dependence affect five test prioritization techniques. Our future work will consist of studying the effects of test dependence on other test prioritization techniques particularly branch coverage and other test techniques such as test selection. 
\item Preventing dependent tests. By preventing developers from writing dependent tests, we can eliminate the impact dependent tests will have on testing techniques. Developers should be encouraged to write tests ''defensively'' by specifying necessary test execution pre-conditions and using less (or properly mocking) global variables or shared resources. There is already some work aiming at automating this process to prevent the potential for dependences by refactoring programs to use less global states [9].
\end{itemize}
\end{comment}
