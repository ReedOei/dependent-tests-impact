
\subsection{Automatically Explaining Test Dependence}
\label{sec:coperoot}


Test dependencies arise largely because of improper access
to shared global variables or external resources (e.g., files).
However, realizing the root causes of test dependence is often no
obvious.
The global variables or external resources 
accessed by two tests are usually buried deep in
the program code, and the assertions do not directly check
them, but rather check values that have been computed from
them. In any non-trivial real-world program, this deep
nesting effectively hides potential dependencies from developers,
and they may only become aware of them when a subtle bug
leads them there. This section presents an automated
technique that generates a concise report to explain the
root cause of such test dependence.


\subsubsection{Explanation Technique}

Our technique, called \dtexplain, contains three steps:

\begin{enumerate}
\item \textbf{Simplification}. For a dependent test $t$,
\dtexplain uses Delta Debugging to isolate the shortest test
execution sequence $T_{seq}$ that if executed before $t$, will lead
$t$ to exhibit a different result than the result of
executing $t$ alone.

\item \textbf{Execution with Instrumentation}. \dtexplain instruments
the tested program, $t$, and all tests in $T_{seq}$ to monitor
the access to shared global variables and files. \dtexplain 
executes $t$ and $T_{seq}$ in two different order. \dtexplain
first executes $t$ alone to observe its accessed variables
and files. Then, \dtexplain executes $T_{seq}$ before $t$
and observes variables and files accessed by both $T_{seq}$
and $t$.

\item \textbf{Summarization}. \dtexplain summarizes the
variable-accessing information to generate a report to describe
which fields $t$ and $T_{seq}$ may have improper access.
If the field access occurs in the library code, \dtexplain
``bubbles up" it to the application code.
\dtexplain also presents the method-call sequence
that leads to the improper field access. The detailed
algorithm for this step is given in Figure~\ref{fig:summarization}.
\end{enumerate}

We next explain the algorithm for the thrid step with
a concrete example in Figure~\ref{fig:dtex}.
